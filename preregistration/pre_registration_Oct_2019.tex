\documentclass[]{article}
\usepackage{lmodern}
\usepackage{amssymb,amsmath}
\usepackage{ifxetex,ifluatex}
\usepackage{fixltx2e} % provides \textsubscript
\ifnum 0\ifxetex 1\fi\ifluatex 1\fi=0 % if pdftex
  \usepackage[T1]{fontenc}
  \usepackage[utf8]{inputenc}
\else % if luatex or xelatex
  \ifxetex
    \usepackage{mathspec}
  \else
    \usepackage{fontspec}
  \fi
  \defaultfontfeatures{Ligatures=TeX,Scale=MatchLowercase}
\fi
% use upquote if available, for straight quotes in verbatim environments
\IfFileExists{upquote.sty}{\usepackage{upquote}}{}
% use microtype if available
\IfFileExists{microtype.sty}{%
\usepackage{microtype}
\UseMicrotypeSet[protrusion]{basicmath} % disable protrusion for tt fonts
}{}
\usepackage[margin=1in]{geometry}
\usepackage{hyperref}
\hypersetup{unicode=true,
            pdftitle={Pre-registration\_Geomphon\_hindi},
            pdfauthor={Amelia},
            pdfborder={0 0 0},
            breaklinks=true}
\urlstyle{same}  % don't use monospace font for urls
\usepackage{graphicx,grffile}
\makeatletter
\def\maxwidth{\ifdim\Gin@nat@width>\linewidth\linewidth\else\Gin@nat@width\fi}
\def\maxheight{\ifdim\Gin@nat@height>\textheight\textheight\else\Gin@nat@height\fi}
\makeatother
% Scale images if necessary, so that they will not overflow the page
% margins by default, and it is still possible to overwrite the defaults
% using explicit options in \includegraphics[width, height, ...]{}
\setkeys{Gin}{width=\maxwidth,height=\maxheight,keepaspectratio}
\IfFileExists{parskip.sty}{%
\usepackage{parskip}
}{% else
\setlength{\parindent}{0pt}
\setlength{\parskip}{6pt plus 2pt minus 1pt}
}
\setlength{\emergencystretch}{3em}  % prevent overfull lines
\providecommand{\tightlist}{%
  \setlength{\itemsep}{0pt}\setlength{\parskip}{0pt}}
\setcounter{secnumdepth}{0}
% Redefines (sub)paragraphs to behave more like sections
\ifx\paragraph\undefined\else
\let\oldparagraph\paragraph
\renewcommand{\paragraph}[1]{\oldparagraph{#1}\mbox{}}
\fi
\ifx\subparagraph\undefined\else
\let\oldsubparagraph\subparagraph
\renewcommand{\subparagraph}[1]{\oldsubparagraph{#1}\mbox{}}
\fi

%%% Use protect on footnotes to avoid problems with footnotes in titles
\let\rmarkdownfootnote\footnote%
\def\footnote{\protect\rmarkdownfootnote}

%%% Change title format to be more compact
\usepackage{titling}

% Create subtitle command for use in maketitle
\providecommand{\subtitle}[1]{
  \posttitle{
    \begin{center}\large#1\end{center}
    }
}

\setlength{\droptitle}{-2em}

  \title{Pre-registration\_Geomphon\_hindi}
    \pretitle{\vspace{\droptitle}\centering\huge}
  \posttitle{\par}
    \author{Amelia}
    \preauthor{\centering\large\emph}
  \postauthor{\par}
      \predate{\centering\large\emph}
  \postdate{\par}
    \date{01/10/2019}


\begin{document}
\maketitle

\hypertarget{letter}{%
\section{Letter}\label{letter}}

The process of perception of an unfamiliar speech sound is theorized to
involve a comparison of the unfamiliar sound to representations of
familiar sounds heard in the past. However, research testing the ability
to discriminate an unfamiliar sound often focuses on the difference
betweeen the unfamiliar sound and one familiar sound that is determined
to be the ``nearest'' sound in perceptual or acoustic space. This narrow
focus on the one nearest sound ignores the rest of a large and often
crowded represetation space.

This is important because it is known that sound inventories of language
are not random in their geometric shape, but instead can be measured to
have global and local symmetry and economy, above and beyond what would
be expected by chance. (Dunbar \& Dupoux 2016) We ask: does the shape of
the entire sound inventory affect discirmination of an unfamiliar sound,
above and beyond the effect of distance to the nearest sound? Our
research has implications for theories of language change and second
language acquisition and practical applications in language teaching.

The study has been approved by the Comité d'Éthique de la Recherche of
Université Paris Decartes as of June 13, 2019. All funding for
participant payment is covered by the Agence National de Récherche grant
numbers ANR-17-CE28-0009 (GEOMPHON) and ANR-10-LABX-0083 (EFL). Web
hosting for the online studies will be provided by the Laboratoire de
Linguistique Formelle. The study is ready to commence immediately.

Data collection for the English speakers will take a maximum of one week
to complete. Based on previous studies we expect recruitment of French
participants will take longer, but we anticipate being able to run all
participants within a two week window. Once data is collected all code
is in place to immediately run models and create vizualizations. At this
point we will take one further week to move from a stage 1 manuscript to
a stage 2 manuscript. The total time anticipated to go from in principle
approval to stage 2 manuscript is one month.

We agree to participate in the Open Peer Review system.

Following Stage 1 in principle acceptance, we agree to register the
approved protocol on the Open Science Framework either publicly or under
private embargo until submission of the Stage 2 manuscript.

If we should withdraw the paper following in principle acceptance, we
agree to the publicaiton of a short summary of the pre-registered study
as a Withdrawn Registration.

appropriate reviewers with expertise in linguistics and Bayesian Methods
include:

Jeff Mielke\\
Timo Roettger\\
Bodo Winter\\
Till Poppels

\hypertarget{introduction}{%
\section{Introduction}\label{introduction}}

\hypertarget{background-and-literature-review}{%
\subsubsection{background and literature
review}\label{background-and-literature-review}}

Over a lifetime of speech perception, humans fine-tune their sensitivity
to phonetic differences that are contrastive in their native language.
When listeners learn a new phone, be it a phone from a new language or a
variant they have not previously encountered, they are faced with the
task of perceiving new sounds with a perceptual system that is tuned to
their native language. There is extensive research investigating
precisely how this is accomplished. In formal models of language
learning such as the Speech Learning Model (Flege) and the Perceptual
Assimilation Model (Best,1995, Best \& Tyler 2007), an important concept
that has arisen from past work is the distance of the unfamiliar sound
to the nearest familiar sound in the native inventory. Unfamiliar sounds
will be hard to discriminate from a familiar sound that is close, and
easier to discriminate from a familiar sound that is farther away.

This ``distance'' is defined in different ways: principally, as either
an objective acoustic distance measured on some element of an acoustic
signal, or as a perceptual distance which can be measured through
discrimination tasks. However, a narrow focus on the distance between
one familiar sound and one unfamiliar sound ignores the crucial fact
that phonetic space is crowded for most sound inventories. An unfamiliar
sound which is ``far'' from one familar sound may be ``near'' to other
sounds, and if a distribution or category develops and changes over
time, a move ``farther'' from one sound is likely to be ``nearer'' to
another sound. We can measure this effect by looking at how the shape of
an inventory would change if a new sound were added.

We argue that perception of a phone involves not just the process of
comparing the new sound to the nearest old sound, but also as the
process of adding new sounds into an already established inventory. In
particular, we are interested in how an inventory would change if the
new sound is incorporated. It has been shown previously that sound
inventories are not random in their distribution of sounds (Dunbar \&
Dupoux 2016). Dunpar \& Doupoux (2016) develop three measurements which
encompass these values:

\begin{enumerate}
\def\labelenumi{\arabic{enumi})}
\item
  economy (``Econ''): if two inventories have the same number of sounds,
  those sounds tend, all things being equal, to require relatively few
  features to characterize
\item
  local symmetry (``Loc'') : if two inventories have the same number of
  sounds, using the same number of features, those sounds tend to make
  relatively many ``minimal oppositions'' between pairs of sounds, which
  differ only in one feature
\item
  global symmetry (``Glob''): if two inventories are matched
  geometrically on the other two properties, they tend to have sounds
  that are more symmetrically distributed
\end{enumerate}

Because these metrics measure the geometry of an entire inventory, we
call these measures ``geomphon scores'' The calculation of these scores
is based on discrete phonological features, and details of the precise
method are available in Dunbar \& Dupoux (2016). While Dunbar \& Dupoux
establish that sound inventories of the worlds languages maximize these
properties above and beyond the distribution expected by chance, they do
not establish \emph{why} inventories maximize these scores. One possible
reason is that inventories that maximize these scores are optimal for
discrimination of sounds within an inventory. This would lead to the
prediction that a change in inventory that reduces one of these scores
would lead to worse discrimination, when compared to a change in
inventory that increases one of these scores.

In this project, we collect discrimination data from an ABX task in
which listeners discriminate pairs of sounds. We use French and English
speakers, listening to unfamiliar Hindi sounds.

\hypertarget{experimental-aims}{%
\subsubsection{Experimental aims}\label{experimental-aims}}

The experiment aims to test whether the spatial relation of all the
sounds in an inventory affect perception of a given unafamiliar sound.
Specifically, we test whether changes in geomphon scores are predictive
of discrimination accuracy, above and beyond the effect of acoustic
distance to one nearest sound.

\hypertarget{hypotheses}{%
\subsubsection{Hypotheses}\label{hypotheses}}

H1) We hypothesize that, as previous research has shown, a pair of
sounds that are farther apart in acoustic distance will be easier to
discriminate than two sounds that are very close in acoustic distance.

H2) We hypothesize that changes in Econ, Glob, and Loc will each be
predictive of discimination accuracy, above and beyond the effect of
acoustic distance.

\hypertarget{method}{%
\section{Method}\label{method}}

\hypertarget{sample}{%
\subsubsection{sample}\label{sample}}

\hypertarget{population}{%
\paragraph{population}\label{population}}

Participants will be recruited in three venues:

\begin{enumerate}
\def\labelenumi{\arabic{enumi})}
\tightlist
\item
  English speaking participants will be recruited on Amazon Mechanical
  Turk, with qualifications set to:

  \begin{itemize}
  \tightlist
  \item
    location is US\\
  \item
    approval rate 95\% or higher\\
  \item
    have completed \textgreater{} 100 HITS\\
  \item
    following mechanical turk's policies, all Amazon Mechanical Turk
    participants are \textgreater{} 18 years of age
  \end{itemize}
\item
  French speaking paricipants will be recruited recruited on Amazon
  Mechanical Turk, with qualifications set to:

  \begin{itemize}
  \tightlist
  \item
    location is France\\
  \item
    approval rate 95\% or higher\\
  \item
    have completed \textgreater{} 100 HITS\\
  \item
    following mechanical turk's policies, all Amazon Mechanical Turk
    participants are \textgreater{} 18 years of age
  \end{itemize}
\item
  French speaking participants will also be recruited through RISC and
  the facebook groups for linguistics and Psychology research.\\
  \textbf{DECISION: will we use these? will these speakers be paid?}
\end{enumerate}

\hypertarget{inclusion-and-exclusion-criteria}{%
\paragraph{Inclusion and exclusion
criteria}\label{inclusion-and-exclusion-criteria}}

Each participant will answer a demographic survey designed to determine
their eligibility. To incentivize truthful behavior in online
participants, all Mechanical Turk participants will be paid regardless
of their answers.

Data from any subject who meets \textbf{any} of the following criteria
will be \textbf{excluded}.

\begin{enumerate}
\def\labelenumi{\arabic{enumi})}
\tightlist
\item
  They answer ``Yes'' to question ``Do you have any problems with your
  hearing or vision, other than glasses?''
\item
  They answer ``Yes'' to question ``Have you been diagnosed with a
  speech or language impairment?''
\item
  They answer ``Yes'' to question ``Have you ever taken a linguistics
  course?''
\item
  \textbf{DECISION: include this?? Their answer to the question ``Do you
  have an idea about what we might have been investigating?'' indicates
  that they were aware of the purporse of the experiment.}
\item
  They indicate they have high proficiency in another language **(They
  answer \_\_\_ to the question \_\_\_)
\item
  they indicate they have high experience in another language (They
  answer \_\_\_ to the question \_\_\_)\\
\item
  they indicate they have both medium proficiency and medium experience
  in another language\\
\item
  **DECISION: INCLUDE THIS?Their performance is not significantly
  different from chance, as determined by\_\_\_**\\
\item
  **DECISION: INCLUDE THIS? Their performance is not significantly
  different from ceiling, as determined by\_\_\_**\\
\item
  \textbf{DECISION: INCLUDE THIS? Participant has NA/no data for
  \textgreater{}2 trials (for any reason, e.g.user error, unanticipated
  technical error)}
\end{enumerate}

We will run a batch of 90 subjects (the total goal number, based on
power analysis below), and then run the above analyses to determine the
number of subjects that pass all criteria. We will then run batches of
size (90 - number who have already passed the above criteria) until we
reach 90 subjects. Each batch of data will be posted to OSF in raw,
anonymized format on the day of collection, before a new batch is
collected.

\hypertarget{number-of-subjects}{%
\paragraph{number of subjects}\label{number-of-subjects}}

\textbf{AMELIA note: (consider adding graph of N based on acoustic
distances from disastrous Roger Levy presentation)}

Our goal was to ensure that for our given design there would be enough
data that the mean of the posterior distribution would not differ
greatly from the true effect. To determine the number of subjects
necessary to achive this goal, we tested directly by sampling datasets
based on a variety of values of coefficients, and then ran models to
detect these known coefficient values.

We graph below the difference between the value input as the coefficient
and the mean of the posterior for 27 combinations of values (All
combinations of 3 coefficient values (-1,0,1), for 3 predictors) . In
the best case scenario the mean of the posterior is exactly the same as
the coefficient value we used to sample the data, and therefore the
difference shown on this chart is zero. We can see that the distribution
approaches zero as the number of subjects increases.

An N of \emph{90 subjects} was selected as the best balance between the
practicalities of running many subjects and the extent to which the mean
of the posterior distribution would reflect the true value.

\hypertarget{stimuli}{%
\subsubsection{stimuli}\label{stimuli}}

\hypertarget{choice-of-unfamiliar-language}{%
\subparagraph{Choice of unfamiliar
language:}\label{choice-of-unfamiliar-language}}

Because this experiment aims to tease apart the effects of acoustic
distance and each of the geomphon scores, it was important to find a
language that has sounds that are both familiar and unfamiliar. For
practical reasons our discrimination experiments are to be run with
English and French listeners, and therefore it was necessary to find a
language which has some sounds that are similar to sounds in the
inventory of French and English, and some sounds that are highly
dissimilar. In addition, the unfamilar sounds should involve the best
spread of each geomphon score, as well as acoustic distances.

While geomphon scores can be calculated based on a full inventory of
consonants and vowels, practically speaking, discrimination between
pairs of sounds consisting of one vowel and one consonant would yield
ceiling level accuracy and be uninformative for our research question.
We therefore use only consonants

Dunbar and Dupoux calculated Geomphon scores for \textbf{N} languages
and the feature matrices used to do this are available \textbf{here}
Using these matrices, we identified a list of languages that
\textbf{criteria}. We then calculated scores base don \textbf{fill in
details from monday meeting}

Of the languages on this list, Hindi was chosen because it was a
language that has many consonants that are familiar to French and
English speakers \textbf{examples}, and also many that are unfamiliar
(in particular, retroflex sounds). It was also a language for which we
would be able to recruit speakers in France and the United States.
\textbf{other reasons we chose Hindi? other details about how this
happened?}

\hypertarget{procedure-for-recording-of-stimuli}{%
\subparagraph{Procedure for recording of
stimuli:}\label{procedure-for-recording-of-stimuli}}

5 Hind-Urdu speakers (3 female) were recorded in a sound attenuated
booth in either the Université de Paris or the University of
Massachusetts. All reported that Hindi was the principal language they
spoke with their family as children. The primary motivation for using
multiple speakers was to introduce variability and noise into the
experiment, and so the speakers were not controlled for sociolinguistic
factors (e.g.~age, place of birth,etc.) Each speaker was paid \$20 for
their recording and signed an informed consent form.

Participants spoke all consonants of Hindi in a pseudoword /a\_a/
context. Each pseudo word was written on a list in Devanagri and latin
script, with a sample word in Devanagri including the target consonant
sound. For elicitation purposes each pseudoword was said in a sentence
context ``kripya \_\_\_\_ kahie'' \textbf{add Devanagri and IPA}
(``please bring \_\_\_\_\_''). Speaker 1 recorded all peseudowords once.
Speakers 2-5 recorded all consonant sounds twice. The first time,
speakers 2-5 listened to speaker 1, and then paused the recording and
repeated after Speaker 1. They were told to try to repeat at about the
same pace. Recordings are considered individually identifiable
information and so are not available publically, but are approved for
sharing for research purposes, including replication, by emailing
\textbf{ewansemail}

Textgrids for all recordings were created using forced alignment
\textbf{details of tools Nicolas used}. Transcripts, code, and textgrids
are available at \textbf{URL} All textgrids were hand checked and
ajusted for accuracy, and all splice points were moved to the nearest
zero crossing.

\hypertarget{procedure-for-building-and-selecting-stimuli}{%
\paragraph{procedure for building and selecting
stimuli}\label{procedure-for-building-and-selecting-stimuli}}

Each stimulus tests one pair of phonemes. One member of the pair is a
familiar sound, and one member is an unfamiliar sound. It is important
to clarify that the ``familiar'' sounds were hindi sounds spoken by the
hindi speakers. However, these were sounds such as /d/ or /z/ that are
included in both French and English inventories. This means that the
``familiar'' sounds may differ systematically from the native versions.
However, since these ``familiar'' sounds are present in the inventories
of French and English (i.e.~would be transcribed as the same phone in
IPA), and the unfamiliar sounds are not, we expect that the familiar
sounds are much closer to native sounds (\textbf{and indeed our
measurements confirm this}). The rationale for performing the experiment
this way was to have consistency in the context surrounding the phone of
interest. If English speakers recorded the english consonants and Hindi
speakers the hindi consonants, the surrounding vowels have the potential
to give additional cues useful for discrimination and push
discrimination towards ceiling, **as indeed we found in previous
pilot\_\_**

The unfamiliar sounds, on the other hand, were sounds which are
unambiguously absent in both French and English, such as as /dʒʰ/, /bʰ/,
or /ɖ,/.

\textbf{stumuli selection section from other rmd goes here}

Once the pairs of phonemes were chosen, it was necessary to determine
the order of the phonemes, which sound would be the target, and which
speaker's recording would be used. The number of trials was constrained
to 150 based on time estimates of previous experiments. Given the number
of permutations of these options it is not feasible to simply
counterbalance all combinations of these within the number of trials,
which was constrained to 150 trials based on .

Instead, we ensure that the answer is not inferrable from the order or
speaker by defining weights for, and use a simulated annealer to
optimize a design in which we minimize the extent to which the speaker
and order is predictive of the correct answer.

\textbf{stimuli are cut out and spliced together LIKE SO}

\hypertarget{calculation-of-geomphon-scores}{%
\subsubsection{calculation of geomphon
scores}\label{calculation-of-geomphon-scores}}

\hypertarget{ewan}{%
\section{\texorpdfstring{\textbf{EWAN}}{EWAN}}\label{ewan}}

\hypertarget{experimental-procedure}{%
\subsubsection{experimental procedure}\label{experimental-procedure}}

Code sufficient to launch a direct replication, including all
demographic surveys, instructions, and stimuli, presented in a docker
enviroment that replicated exactly the software versions, libraries, and
modules we used, can be found here:

\_\_\_PERSISTENT URL.

A description of method is below:

Participants first read a consent form and indicate their consent. Then
participants fill out a demographic survey.

Participants hear a triplet

randomization, training, instructions, etc.

\hypertarget{model-specifications}{%
\subsubsection{model specifications}\label{model-specifications}}

\hypertarget{calculation-and-choice-of-covariates}{%
\paragraph{calculation and choice of
covariates}\label{calculation-and-choice-of-covariates}}

\textbf{this is in the pre-prereg\_rmd} acoustic distance -- how chose
how calculated MFCCs? log distance? delta distance? \emph{needed: review
of exactly how acoustic distances are calculated, as well as motivation
for a specific acoustic distance score} \emph{some info in old code ver
sept 2018 }

geomphon scores -caculated based on \_\_\_\_

\hypertarget{method-to-determine-whether-an-effect-of-a-coefficient-is-non-zero}{%
\paragraph{method to determine whether an effect of a coefficient is
non-zero}\label{method-to-determine-whether-an-effect-of-a-coefficient-is-non-zero}}

Based on our design we expect acoustic distance will be non-zero.
(indeed, we have specifically designed the stimuli to ensure this is the
case)

The effect of the geomphon scores, however, is as yet undetermined, and
is expected to be smaller than the effect of acoustic distance \emph{?}
Therefore, a significant concern was to determine a model that would
detect a difference between a small, positive effect and a zero effect.

To this end, we use comparison of leave-one-out-cross valitation
(``loo'') scores with models specified with varying priors. The loo
value obtained is a measure of the predictive ability of a model.
Specifically, it is an estimate out-of-sample predictive accuracy using
within-sample fits. We use the pareto smoothed importance sampling
implemented in the loo() package, as documented in Vehtari et al.
(2016).

Because we cannot be sure of the size of the expected effect, it is
important to establish a methodology that is sensitive to small but
non-zero positive effects. To accomplish this we use loo() and truncated
priors. A truncated prior is simply a prior for which the density curve
is truncated at some value.

In our case, we use a normal(0,10) prior and truncate the value at zero.
We can therefore apply a model truncated to be \textless{}= 0 to a model
to be \textgreater{}= 0. If an effect of interest is not meaningfully
different than zero, two models, one with a truncated positive prior and
one with a truncated negative prior will not differ (nor would they
differ from a model with no effect of this predictor). If, on the other
hand, the effect is different from zero, either the truncated positive
prior or the truncated negative prior will show better predictivity in
the direction of the effect.

For example, to test for an effect of Econ, we will run a model such as
the following:

accuracy \textasciitilde{} Econ + Glob + Loc + acoustic\_distance +
(1\textbar{}Subject) + (1\textbar{}phone\_pair)

we will run the model three times:

\begin{itemize}
\item
  Model A: removing the effect of Econ, (simulating a zero effect)\\
  accuracy \textasciitilde{} Glob + Loc + acoustic\_distance +
  (1\textbar{}Subject) + (1\textbar{}phone\_pair),\\
  priors: N(0,10) for all covariates
\item
  Model B applying a prior truncated to be \textgreater{}=0\\
  accuracy \textasciitilde{} Econ + Glob + Loc + acoustic\_distance +
  (1\textbar{}Subject) + (1\textbar{}phone\_pair),\\
  prior Econ: truncated gaussian N(0,10) \textgreater{}=0 all other
  priors: N(0,10)
\item
  Model C applying a prior trunacted to be \textless{}=0 accuracy
  \textasciitilde{} Econ + Glob + Loc + acoustic\_distance +
  (1\textbar{}Subject) + (1\textbar{}phone\_pair) prior Econ: truncated
  gaussian N(0,10) \textless{}=0 all other priors: N(0,10)
\end{itemize}

We will then complete pairwise differences to compare the differences in
loo scores between each of these four using loo\_compare(), which gives
a we used the difference in expected log predictive density
(elpd\_diff). The elpd\_diff shows which model has a better loo score,
and therefore better predictive accuracy. The output of loo.compare()
includes the s.e. of this difference score. We will determine whether
two models is meaningfully ``different'' by using the rule of thumb put
forth by Vehtari that the elpd\_diff is 5 times larger than it's
s.e.\emph{?}

In order to make all comparisons of all combinations of the three
predictors of interest, we will run a total of 27 models (all
combinations of 3 predictors (Econ, Glob, Loc) and three priors
(truncated posistive, truncated negative, zero)).

We will then use compare\_models() in the loo package to obtain a
ranking of their elpd\_diff scores.

We will interpret there to be a positive effect of one of the geomphon
scores Model B is better than model A AND model C.

\hypertarget{results-analysis}{%
\section{Results \& Analysis}\label{results-analysis}}

\hypertarget{preprocessing}{%
\subsubsection{Preprocessing}\label{preprocessing}}

Data will be filtered for whether it confirms to criteria outlined in
subejct inclusion/exclusion above. All data that meets criteria will be
used in analysis.

\textbf{While we have no reason to antifcipate messy data or NAs, Here
is our policy}

\hypertarget{reality-checks}{%
\subsubsection{reality checks}\label{reality-checks}}

Our hypotheses take for granted an effect of acoustic distance.
Therefore, an effect of acoustic distance is a required first step
before interpreting the geomphon scores.(see interpretive plan, below)

\hypertarget{models-any-covariates-or-regressors-must-be-stated.}{%
\subsubsection{models Any covariates or regressors must be
stated.}\label{models-any-covariates-or-regressors-must-be-stated.}}

\emph{Consistent with the guidelines of Simmons et al. (2011), proposed
analyses involving covariates must be reported with and without the
covariate(s) included.}

All covariates:\\
acoustic distance

We will determine whether there is an effe

We will It is not clear on what scale this should be.\\
previous pilots show resutls on the order of \_\_\_\_

\hypertarget{interperative-plan}{%
\section{Interperative plan}\label{interperative-plan}}

Our interpretive plan therefore will be

step 1: confirm effect of acoustic distance.

possible outcomes:

\begin{enumerate}
\def\labelenumi{\alph{enumi})}
\tightlist
\item
  there is no effect of acoustic distance
\end{enumerate}

We will interpret this as a failure to replicate the well-known effect
of acoustic distance. Given that the stimuli were chosen specifically to
have an effect of acoustic distance, that the number of subjects was
chosen specifically to be able to robustly detect this effect, and that
three previous pilots show large significant effects of distance, we
will we will consider this a fatal failure of the experimental protocol.
We will withdraw the study and pre-register a new experimental protocol
with more subjects and/ or an easier task in order to detect the effect
of acoustic distance.

\begin{enumerate}
\def\labelenumi{\alph{enumi})}
\setcounter{enumi}{1}
\tightlist
\item
  there is an effect of acoustic distance
\end{enumerate}

We will interpret this to be a confirmation that our experimental
protocol was well-designed to detect the effects of interest.

step 2: examine the effect of geomphon scores

possible outcomes:

\begin{enumerate}
\def\labelenumi{\alph{enumi})}
\item
  there is an effect of one or more of the geomphon features, above an
  beyond the effect of acoustic distance only.
\item
  there is no effect of any of the geomphon scores above and beyond the
  effect of acoustic distance.
\end{enumerate}

\hypertarget{timeline}{%
\section{timeline}\label{timeline}}

Timeline for completion of the study and proposed resubmission date if
registration review is successful. Extensions to this deadline can be
negotiated with the action editor.


\end{document}
